\documentclass{beamer}
\usepackage[utf8]{inputenc}
\usepackage{color}
\usepackage{subcaption}
\usepackage{tikz}
\usetikzlibrary{graphs,positioning}
\usepackage{lmodern}

\usepackage{amsmath,amssymb,amsthm}    
\usepackage{mathabx}\changenotsign   
\usepackage{mathrsfs} 
\usepackage{dsfont} 
\usepackage[babel]{microtype}
\usepackage{xcolor}  	
\usepackage[export]{adjustbox}

\usepackage[brazil]{babel}   
\usepackage[utf8]{inputenc} 
\usepackage{comment}

\newtheorem{thm}[equation]{Teorema}
\newtheorem{cor}[equation]{Corolário}
\newtheorem{lem}[equation]{Lema}
\newtheorem{prop}[equation]{Proposição}
\newtheorem{conj}[equation]{Conjectura}


\def\moverlay{\mathpalette\mov@rlay}
\def\mov@rlay#1#2{\leavevmode\vtop{   \baselineskip\z@skip \lineskiplimit-\maxdimen
		\ialign{\hfil$\m@th#1##$\hfil\cr#2\crcr}}}
\newcommand{\charfusion}[3][\mathord]{
	#1{\ifx#1\mathop\vphantom{#2}\fi
		\mathpalette\mov@rlay{#2\cr#3}
	}
	\ifx#1\mathop\expandafter\displaylimits\fi}
\makeatother

\DeclareMathOperator{\dom}{{\rm dom}}

\newcommand{\dcup}{\charfusion[\mathbin]{\cup}{\cdot}}
\newcommand{\bigdcup}{\charfusion[\mathop]{\bigcup}{\cdot}}


\def\ra{\longrightarrow}
\def\dom{\text{\rm dom}}

\def\R{\text{\textcolor{red}{\rm red}}}
\def\B{\text{\textcolor{blue}{\rm blue}}}


\def\mcarrow{\xrightarrow[{\raisebox{.5mm}[1mm][0mm]{$\scriptstyle \rm p$}}]{\raisebox{0.0mm}[0mm]{$\scriptstyle \rm mc$}}}
\def\pmc#1{p^{\rm mc}_{#1}}

\def\red{\text{\textcolor{red}{\rm red}}}
\def\blue{\text{\textcolor{blue}{\rm blue}}}
\def\green{\text{\textcolor{gree}{\rm green}}}

\tikzset{onslide/.code args={#1#2}{%
    \only<#1>{\pgfkeysalso{#2}}
}}

\newtheorem{teorema}             {Teorema}       
\newtheorem{Afirmativa}[teorema] {Claim}         
\newtheorem{lema}      [teorema] {Lema}         
\newtheorem{corolario} [teorema] {Corolário}     
\newtheorem{fato}      [teorema] {Fato}    
\newtheorem{proposicao}      [teorema] {Proposição}                
\newtheorem{conjectura}[teorema] {Conjectura}    
\newtheorem{problema}  [teorema] {Problema}       
\newtheorem{definicao}  [teorema] {Definição}       









\title[Combinatória Extremal]{PGC \\ Combinatória Extremal}

\author[Diogo Alves - UFABC]{\textbf{Diogo Eduardo Lima Alves}\\ \ \\ \ \\ \ \\ \ \\Universidade Federal do ABC - UFABC\\ \ \\ \ \\ 
}

\date{25 de junho de 2019}

\usetheme{Madrid}
\usecolortheme{beaver}


\setbeamertemplate{part page}{
        \begin{beamercolorbox}[sep=15pt,center,wd=\textwidth]{part title}
            \usebeamerfont{part title}\insertpart\par
        \end{beamercolorbox}
}


\begin{document}
\maketitle

\part{Combinatória Extremal}

\frame{\partpage}

\frame{
	\frametitle{Combinatória Extremal}

	\begin{itemize}
	\item		\textbf{Combinatória Extremal}: Subtema da combinatória  que estuda quão grande ou pequena uma estrutura pode ser ao mesmo tempo que satisfaz certas condições.\vspace{0.2cm}
	\end{itemize}
}

\frame{
	\frametitle{Combinatória Extremal}

	Exemplos:\vspace{0.2cm}
	\begin{itemize}
		\item Qual a maior quantidade de arestas que um grafo $G$ pode ter, sem que $G$ tenha um subgrafo $H$?\vspace{0.2cm}\pause
		
		\item Qual o tamanho do maior conjunto independente em um grafo? (NP-completo)					
	\end{itemize}
	
}

\frame{
	\frametitle{Mantel}
	
	\begin{teorema}[{Mantel, 1907}]\label{the:mantel}
Se $G$ \'e um grafo livre de tri\^angulos com $n$ v\'ertices, ent\~ao
$$e(G) \leq \left\lceil \frac{n}{2} \right\rceil \left\lfloor \frac{n}{2} \right\rfloor .$$
\end{teorema}
}

\frame{
		\frametitle{Mantel}
\begin{proof}
A prova segue por indu\c{c}\~ao em $n$. Assuma que $n \geq 4$ e $G'$ \'e o grafo obtido de $G$ pela remo\c{c}\~ao de dois v\'ertices $u,v \in G$ tal que $\{u,v\}\in E(G)$. Note que se $G$ \'e tri\^angulo livre, ent\~ao $G'$ tamb\'em \'e tri\^angulo livre porque n\~ao \'e poss\'ivel formar um tri\^angulo removendo uma aresta. Note que existem no m\'aximo $n-1$ arestas incidentes com $u$ ou $v$, ent\~ao
\begin{align*}
e(G') &\leq \left(\left\lfloor \frac{n}{2} \right\rfloor -1\right)\left(\left\lceil \frac{n}{2} \right\rceil -1\right) \\
&= \left\lfloor \frac{n}{2} \right\rfloor \left\lceil \frac{n}{2} \right\rceil - n +1.
\end{align*}	
Portanto, $e(G) \leq e(G') + n-1 = \left\lfloor \frac{n}{2} \right\rfloor \left\lceil \frac{n}{2} \right\rceil$. 
\end{proof}
}

\frame{
	\frametitle{\'Areas Abordadas}

	\begin{itemize}
	\item Teoria de Ramsey
	\item Grafos Extremais
	\item Grafos Aleat\'orios
	\item Regularidade
	\end{itemize}
}

\frame{
	\frametitle{Teoria de Ramsey}
	\begin{teorema}[{Ramsey, 1929}]
	Sejam $s, t \geq 2$. Existe um inteiro positivo $R = R(s, t)$ tal que toda a colora\c{c}\~ao de arestas de $K_R$, com as cores vermelho e azul, admite um subgrafo $K_s$ vermelho ou um
subgrafo $K_t$ azul.
\end{teorema}
	\begin{teorema}[{Schur, 1916}]\label{thm:Schur'sTheorem} %Schur, 1916
Para $r = 2$ existe $n \geq 3$ tal que para toda $r$-colora\c{c}\~ao $c\colon \{1, \ldots, n\} \rightarrow [r]$, existem $x,y,z$ tal que $x+y=z$ e $c(x) = c(y) = c(z)$.
\end{teorema}
}
\frame{
	\frametitle{Teoria de Ramsey}
\begin{proof}
  Para toda colora\c{c}\~ao $c\colon [n] \rightarrow [r]$ defina uma colora\c{c}\~ao das arestas de $K_n$ dada por $c'\colon \binom{[n]}{2} \rightarrow [r]$ como $c'(\{a,b\}) \colon= c(|a-b|)$. Pelo Teorema de Ramsey, sabemos que existe um tri\^angulo monocrom\'atico em $K_n$. Assuma que $\{x,y,z\}$ forma um tri\^angulo monocrom\'atico, com $x<y<z$.\\
Usando a definic\~ao de $c'$,  temos:
$$c'(\{x,y\}) = i = c(|y-x|)$$
$$c'(\{x,z\}) =  i = c(|z-x|)$$
$$c'(\{y,z\}) = i = c(|z-y|).$$

Segue que $c(|y-x|) = c(|z-x|) = c(|z-y|)$, e $(z-y)+(y-x)=(z-x)$, logo existe $a+b=c$ com $c(a) = c(b) = c(c)$, como requisitado.\\
\end{proof}
}

\frame{
	\frametitle{Grafos Extremais}
		\begin{teorema}[{Erd\H{o}s, 1938}] \label{theorem: Erdos,1938}

Para todo grafo $G$ com $n$ v\'ertices e livre de $C_4$ temos,

$$e(G) = O(n^{3/2}).$$
\end{teorema}
O $C_4$ \'e formado por duas `cerejas' no mesmo par de v\'ertices. Contando essas triplas (x,y,z) em $G$ tal que $xy, xz \in E(G)$. A Desigualdade de Jensen com $\lambda_i = 1/n$, fornece:
	$$ \sum_{i=1}^n \frac{1}{n} f\left(x_i\right) \geq f		\left(\sum_{i=1}^n \frac{1}{n} x_i\right).$$
}
	
\frame{
	\frametitle{Grafos Extremais}
Aplicando essa desigualdade \`a nossa fun\c{c}\~ao temos,
$$ \frac{\sum_{i=1}^n \binom{x_i}{2}}{n} \geq \binom{\frac{\sum_{i=1}^n x_i}{n}}{2} ,$$
Substituindo $x_i$ por $d(v)$ e lembrando que $\sum_{v \in V(G)} d(v) = 2e(G),$
\begin{align*}
\sum_{v \in V(G)} \binom{d(v)}{2} &\geq n \binom{\frac{2e(G)}{n}}{2}\\
&= n\frac{\frac{2e(G)}{n}\left( \frac{2e(G)}{n}-1\right)}{2} \\
&\geq \frac{n}{2} \left( \frac{2e(G)}{n} - 1 \right)^2.
\end{align*}
}

\frame{
	\frametitle{Grafos Extremais}
Note que o n\'umero dessas triplas em um grafo $C_4$-livre \'e no m\'aximo $\binom{n}{2}$ porque podemos ter apenas uma cereja em cada par de v\'ertices.
$$ \frac{n}{2}\left(\frac{2e(G)}{n} - 1\right)^2 \leq \binom{n}{2},$$
Portanto, $e(G) = O(n^{3/2})$, terminando a prova.
}

\frame{
	\frametitle{Grafos Aleat\'orios}

		\begin{teorema}[Desigualdade de Chebyshev] Seja $X$ uma vari\'avel aleat\'oria com valor esperado $\mu$ e vari\^ancia finita. Ent\~ao para todo $a > 0$,
		$$\mathbb{P}(|X-\mu| 	\geq a) \leq \sigma^2/a^2.$$
		\end{teorema}
		Usando $a = \mu$, temos a seguinte inequac\~ao,

		$$\mathbb{P}(X=0) \leq \mathbb{P}(|X-\mu | \geq \mu) 		\leq \sigma^2/\mu^2,$$
		que nos d\'a um limitante superior para $\mathbb{P}(X=0)$.\\
	
}
\frame{
	\frametitle{Grafos Aleat\'orios}
	\begin{teorema}
		Seja $G=G(n,p)$ um grafo aleat\'orio (Bernoulli).
		Ent\~ao,
		$$
		\mathbb{P}(G\text{ conter um tri\^angulo}) \rightarrow 
		\begin{cases}
			0, &\text{if $p\ll 1/n$},\\
			1, &\text{if $p\gg 1/n$}\,.
		\end{cases}
		$$
	\end{teorema}
	\begin{itemize}
	\item Prova:
	$$X = \text{ quantidade de tri\^angulos em }G$$
Usando Markov,	
	\begin{align*}
	\mathbb{P}(G\text{ conter um tri\^angulo}) &\leq \mathbb{E}(X)\\
	& = \binom{n}{3}p^3 \\
	& \leq n^3p^3\\
	& \ll 1 \text{, se $p \ll 1/n$.},
\end{align*}
	
\begin{align*}
\end{align*} 
	\end{itemize}			
}

\frame{
	\frametitle{Grafos Aleat\'orios}
\begin{align*}
Var(X) &= \mathbb{E}(X^2) - \mathbb{E}(X)^2\\
& = \sum_{u,v} \left(\mathbb{P}(u \wedge v) - \mathbb{P}(u)\mathbb{P}(v)\right)\\
& \leq (n^4p^5-n^6p^6) + (n^5p^6-n^6p^6) + (n^6p^6 -n^6p^6) + (n^3p^3-n^6p^6)\\
& \leq n^4p^5 + n^3p^3,
\end{align*}
usando a Desigualdade de Chebychev,
$$\mathbb{P}(G \text{ n\~ao conter tri\^angulos}) \leq \frac{Var(X)}{\mathbb{E}(X)^2} \leq \frac{n^4p^5 + n^3p^3 }{(n^6p^6/12^2)} \ll 1,$$
para $p \gg 1/n$, terminando a prova.
}

\frame{
	\frametitle{Regularidade}
	\begin{definicao}
Seja um grafo $G$ e $A$, $B$ conjuntos disjuntos de v\'ertices, ent\~ao $(A,B)$ \'e $\varepsilon$-{\it regular} se$\text{ para todo } X \subset A \text{ e } Y \subset B\text{ com } |X| \geq \varepsilon|A| \text{ e } |Y| \geq \varepsilon |B|$ segue
$$ \left| \frac{e(X,Y)}{|X||Y|} - \frac{e(A,B)}{|A||B|} \right| \leq \varepsilon  .$$
\end{definicao}
	
}

\frame{
	\frametitle{Regularidade}
\begin{teorema}[{Szemerédi - Lema da Regularidade, 1975}] Sejam $\varepsilon > 0$ e $m \in \mathbb{N}$. Existem constantes $M=M(m, \varepsilon)$ e $n_0 = n_0(m, \varepsilon)$, tais que, para qualquer grafo $G$ com pelo menos $n_0$ v\'ertices, existe uma parti\c{c}\~ao $V(G) = \{V_0 \cup ... \cup V_k\}$ do conjunto de v\'ertices em $k+1$ classes com $m \leq k \leq M$, onde temos:
\begin{itemize}
	\item $|V_0| \leq \varepsilon |V(G)|$,
	
	\item $|V_1| = \ldots = |V_k|$,
%$|A_0| \leq \varepsilon |V(G)|$,
	\item todos os pares s\~ao regulares com exce\c{c}\~ao de no m\'aximo $\varepsilon k^2$ pares. 
\end{itemize}
\end{teorema}
}

\frame{
	\frametitle{Regularidade}

\begin{lema}\label{lemma:embeddinglemma}
(Lema de Imers\~ao - vers\~ao simplificada) Sejam $H$ e $G$ grafos e $\delta > \varepsilon >0$. Ent\~ao existe $M \in \mathbb{N}$ tal que se $m \geq M$ e existe uma parti\c{c}\~ao de $G$ em $\{V_1, \ldots, V_{V(H)}\}$ com $|V_i| = \lfloor m \rfloor$ ou $|V_i| = \lceil m \rceil$ para $1 \leq i \leq V(H)$ e todos os pares $(V_i,V_j)$ sendo $\varepsilon$-regulares e $\delta$-densos, ent\~ao $H \subset G$. 
\end{lema}
}

\frame{
\begin{teorema}[{Lema da Remo\c{c}\~ao de tri\^angulos}]
Para todo $\alpha > 0$ existe $\beta > 0$ tal que se $G$ \'e um grafo com no m\'aximo $ \beta n^3$ tri\^angulos, ent\~ao \'e poss\'ivel remover todos os tri\^angulos removendo no m\'aximo $\alpha n^2$ arestas. 
\end{teorema}

	\frametitle{Regularidade}
Seguindo a estrat\'egia abaixo os tr\^es primeiros passos s\~ao sempre os mesmos para problemas cl\'assicos.

Prova:

1. Aplicar o Lema da Regularidade com $\varepsilon = 1/k$, $\delta > \varepsilon$ e $m =1/\varepsilon$ ent\~ao obtemos a parti\c{c}\~ao $\{V_0, ... , V_k\}$.

2. Remover arestas  dentro das partes, entre pares irregulares e pares sparsos obtendo $G'$ com no m\'aximo $\alpha n^2$ arestas removidas.

3. Defina $R$ com $V(R) = [k]$ e $\{i,j\} \in E(R) $ se o par $(A_i, A_j)$ \'e $\delta$-denso e $\varepsilon$-regular.
}

\frame{
	\frametitle{Regularidade}
	
H\'a dois casos, no primeiro temos um tri\^angulo em $R$. Note que se $\beta n^3 < 1$ o resultado \'e trivial, ent\~ao assuma que $\beta \geq 1/n^3$ e sobre as partes que formam o tri\^angulo assuma $\{V_1, V_2, V_3\}$  com tamanho $n/k \geq 1/(\beta ^{1/3} k) \geq m$.

Aplicando o Lema de Imers\~ao temos que a quantidade de tri\^angulos em $G'$ \'e no m\'inimo $\delta ^3 /2 (n/k)^3 > \beta n^3$ se escolhermos $\beta$ tal que $\delta ^3 / (2k^3) > \beta$.

Podemos concluir que se $\alpha n^2$ arestas s\~ao removidas e o grafo ainda tem tri\^angulos ent\~ao a quantidade de tri\^angulos \'e maior que $\beta n^3$.

}

\frame{	\frametitle{Regularidade}
No segundo caso n\~ao h\'a nenhum tri\^angulo em $R$ ent\~ao n\~ao temos nenhum tri\^angulo em $G'$ porque as arestas dentro das classes $\{V_1, ... , V_k\}$ foram removidas, encerrando a prova.
}


\frame{
	\frametitle{Agradecimentos}
	\begin{center}
	Obrigado!
	\end{center}
}	
\end{document}
